hello

Outline 
Over the course of the semester, we will cover a wide range of topics from basic economics to core machine learning concepts. We hope that this course will serve as a solid foundation for your future interest in Quantitative Finance. 
This final project will be your chance to delve deeper into a topic you found intriguing or test your own strategy! For example, you can look further into trading concepts like fair values and bet-sizing and research topics like statistical arbitrage and feature engineering to come up with a strategy and test it using skills you learned in lab 1. Expect to put in around 3-6 hours for the project. 
You can form groups of 1-4 and devise a strategy or a model from a topic of your choice. We will not designate specific topics as we encourage students to pursue their own interests and research more about the field. 
Examples: Multifactor Model, Statistical Arbitrage, Asset Valuation, Portfolio Optimization, Pairs Trading, etc 
Guideline 
For strategies, you have three choices. 
1. Implement an academic paper
2. Implement an existing well-known strategy with your own twist. 
3. Model your own strategy 

Traditionally, final projects were done via a Python notebook to implement a strategy and backtest it, and we still recommend this for the best experience, but recently we have broadened the definition of strategy. Some strategies are less able to be tested in a Python notebook, so you’re also allowed to do further research on a quant finance topic online or in a book and report what you learned in a slideshow or paper.
For a backtest-able strategy, it should ideally perform better than the index in terms of returns or Sharpe ratio, and it should be tested and validated with an out-of-sample testing set. 
However, note that finding out that an idea or implementation doesn’t work is a completely valid result. Research often leads to dead ends and we do not encourage you to p-hack or otherwise manipulate your experiment to manufacture a ‘significant’ result. 
You can choose any trading frequency: intra-day, daily, quarterly, etc 
Outline 
Below is a suggested structure for your final project: 
1. Abstract 
2. Introduction 
3. Dataset 
4. Methods (modeling) 
5. Results 
6. Conclusion 
Format
Each group can choose to either write a 3 or more page paper or record an 4-10 minute presentation detailing the results of the project. If you choose the paper, you can follow the outline listed above. If you choose the presentation, you can make it with google slides with detailed explanations of their project from beginning to end or submit a paper detailing the project. 
Questions to consider: 
- Presentation 
- Did you cover your progress from beginning to end? 
- Did you intuitively explain the models? 
- Are there sufficient examples for people to follow? 
- iPython Notebook/Paper
- Does it have an Abstract, Introduction, Body, and Conclusion? 
- Are the graphs and diagrams labeled? 
